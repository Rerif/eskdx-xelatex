% !TeX program = xelatex

%%% Загружаем заголовочный файл, который хранит все настройки и все
%%% подгружаемые пакеты
\newcommand{\No}{\textnumero}

%%% Здесь выбираются необходимые графы
\documentclass[russian,utf8,pointsection,nocolumnsxix,nocolumnxxxi,nocolumnxxxii]{eskdtext}


%%% Что бы работал eskdx и некоторые другие пакеты LaTeX
\usepackage{xecyr}

%%% Для работы шрифтов
\usepackage{xunicode,xltxtra}
%%%% Ставим Times New Roman - как основной шрифт
%\setmainfont[Mapping=tex-text]{DejaVuSerif}
%%%% Courier New - для моноширного текста
%\setmonofont[Scale=MatchLowercase]{DejaVuSansMono}


\usepackage{fontspec}
%%% Для работы с русскими текстами (расстановки переносов)
\usepackage{polyglossia}
\setdefaultlanguage{russian}

%\newfontfamily{\cyrillicfontt}{GOST_B}
%\set{GOST_type_A}
\setmainfont{GOST_type_A}
\setsansfont{GOST_type_A}
\setmonofont{GOST_type_A}
%\newfontfamily\russianfont

\newfontfamily{\cyrillicfont}{GOST_type_A}
\newfontfamily{\cyrillicfontt}{GOST_type_A}
\newfontfamily{\cyrillicfonttt}{GOST_type_A}

\defaultfontfeatures{Mapping=tex-text}
%\defaultfontfeatures{Scale=MatchLowercase} %--несобирается

\setkeys{russian}{babelshorthands=true}


%%% Для того чтобы работали стандартные сочетания символов ---, --, << >> и т.п.
%\defaultfontfeatures{Mapping=tex-text}

%%% Для работы со сложными формулами
\usepackage{amsmath}
\usepackage{amssymb}

%%% Что бы использовать символ градуса (°) - \degree
\usepackage{gensymb}


%%% Для переноса составных слов
%\XeTeXinterchartokenstate=1
\XeTeXcharclass `\- 24
\XeTeXinterchartoks 24 0 ={\hskip\z@skip}
\XeTeXinterchartoks 0 24 ={\nobreak}

%%% Ставим подпись к рисункам. Вместо «рис. 1» будет «Рисунок 1»
\addto{\captionsrussian}{\renewcommand{\figurename}{Рисунок}}
%%% Убираем точки после цифр в заголовках
\def\russian@capsformat{%
  \def\postchapter{\@aftersepkern}%
  \def\postsection{\@aftersepkern}%
  \def\postsubsection{\@aftersepkern}%
  \def\postsubsubsection{\@aftersepkern}%
  \def\postparagraph{\@aftersepkern}%
  \def\postsubparagraph{\@aftersepkern}%
}



% Автоматически переносить на след. строку слова которые не убираются
% в строке
\sloppy

%%% Для вставки рисунков
\usepackage{graphicx}

%%% Для вставки интернет ссылок, полезно в библиографии
\usepackage{url}

%%% Подподразделы(\subsubsection) не выводим в содержании
\setcounter{tocdepth}{2}

%%% Каждый раздел (\section) с новой страницы
\let\stdsection\section
\renewcommand\section{\newpage\stdsection}

%%% В введении нумерация подразделов идёт с буквой «В» (например В.1)
\makeatletter
\renewcommand\thesubsection{\ifnum\c@section=0{В.\arabic{subsection}}\else{\arabic{section}.\arabic{subsection}}\fi}
\makeatother


%%% Загружаем настройки пакета eskdx, там нужно заполнить информацию
%%% о документе - ФИО авторов, название документов и т.п.
%%% Название документа
\ESKDtitle{ ESKDx }
\ESKDdocName{ Заготовка }

\ESKDauthor{ Автор~И.~О. }
\ESKDchecker{ Пров.~И.~О. }
\ESKDnormContr{ Н.Кнтр.~И.О. }

%%% Для титульника
\ESKDtitleApprovedBy{ Должность утверждающего }{ Фам. утвер. }
\ESKDtitleAgreedBy{ Должность первого согласовавшего }{ Фам. первого согл. }
\ESKDtitleAgreedBy{ Должность второго согласовавшего }{ Фам. второго согл. }
\ESKDtitleAgreedBy{ Должность третьего согласовавшего }{ Фам. третьего согл. }
\ESKDtitleDesignedBy{ Должность первого автора }{ Фам. первого автора }
\ESKDtitleDesignedBy{ Должность второго автора }{ Фам. второго автора }

\ESKDdepartment{ Ведомство }
\ESKDcompany{ Предприятие }
\ESKDclassCode{ Код по классификатору }
\ESKDsignature{ Обозначение документа }

\ESKDdate{ 2011/06/13 }


\begin{document}

%%% Делаем титульник
\maketitle

%%% Делаем содержание
\tableofcontents

%%% Введение пишется без цифры и добавляется в содержание
\section*{Введение}
\addcontentsline{toc}{section}{Введение}
Здесь текст введения.

Проснувшись <<однажды утром>> после беспокойного сна, Грегор Замза обнаружил, что он у себя в постели превратился в страшное насекомое. 
Лежа на панцирнотвердой спине, он видел,

\begin{enumerate}
   \item стоило ему приподнять голову, 
   \item свой 
   \begin{enumerate}
      \item коричневый, 
      \item выпуклый, 
      \item разделенный дугообразными чешуйками живот, 
   \end{enumerate}
   
   на верхушке которого еле держалось готовое вот-вот окончательно сползти одеяло. 
   
\end{enumerate}

Его многочисленные, убого тонкие по сравнению с остальным телом ножки беспомощно копошились у него перед глазами. «Что со мной случилось?» – подумал он. Это не было сном. Его комната, настоящая, разве что слишком маленькая, но обычная комната, мирно покоилась в своих четырех хорошо знакомых стенах. Над столом, где были разложены распакованные образцы сукон – Замза был коммивояжером, – висел портрет, который он недавно вырезал из иллюстрированного журнала и вставил в красивую золоченую рамку. На портрете была изображена дама в меховой шляпе и боа, она сидела очень прямо и протягивала зрителю тяжелую меховую муфту, в которой целиком исчезала ее рука. Затем взгляд Грегора устремился в окно, и пасмурная погода – слышно было, как по жести подоконника стучат капли дождя – привела его и вовсе в грустное настроение. «Хорошо бы еще немного поспать и забыть всю эту чепуху», – подумал он, но это было совершенно неосуществимо, он привык спать на правом боку, а в теперешнем своем





\section{Заголовок}
Здесь основной текст работы\cite{BookRef, ArticleRef, LinkRef}.

%%% Заключение, так же как и введение выводим без цифры, добавляем в содержание
\section*{Заключение}
\addcontentsline{toc}{section}{Заключение}
Здесь текст заключения

%%% Далее выводим библиографию
%%% Исправляем ошибку библиографии - «кавычка перед тире»
%%% http://ru-tex.livejournal.com/105178.html?thread=801498
\catcode`"\active\def"{\relax}
\bibliographystyle{gost780s}
\bibliography{bibliography}{}
%%% Если пропадают инициалы, смотрите сюда:
%%% http://plumbum-blog.blogspot.com/2010/10/bibtex-miktex-gost780s.html
\end{document}
