%%% Здесь выбираются необходимые графы
\documentclass[russian,utf8,pointsection,nocolumnsxix,nocolumnxxxi,nocolumnxxxii]{eskdtext}


%%% Что бы работал eskdx и некоторые другие пакеты LaTeX
\usepackage{xecyr}

%%% Для работы шрифтов
\usepackage{xunicode,xltxtra}
%%%% Ставим Times New Roman - как основной шрифт
%\setmainfont[Mapping=tex-text]{DejaVuSerif}
%%%% Courier New - для моноширного текста
%\setmonofont[Scale=MatchLowercase]{DejaVuSansMono}


\usepackage{fontspec}
%%% Для работы с русскими текстами (расстановки переносов)
\usepackage{polyglossia}
\setdefaultlanguage{russian}

%\newfontfamily{\cyrillicfontt}{GOST_B}
%\set{GOST_type_A}
\setmainfont{GOST_type_A}
\setsansfont{GOST_type_A}
\setmonofont{GOST_type_A}
%\newfontfamily\russianfont

\newfontfamily{\cyrillicfont}{GOST_type_A}
\newfontfamily{\cyrillicfontt}{GOST_type_A}
\newfontfamily{\cyrillicfonttt}{GOST_type_A}

\defaultfontfeatures{Mapping=tex-text}
%\defaultfontfeatures{Scale=MatchLowercase} %--несобирается

\setkeys{russian}{babelshorthands=true}


%%% Для того чтобы работали стандартные сочетания символов ---, --, << >> и т.п.
%\defaultfontfeatures{Mapping=tex-text}

%%% Для работы со сложными формулами
\usepackage{amsmath}
\usepackage{amssymb}

%%% Что бы использовать символ градуса (°) - \degree
\usepackage{gensymb}


%%% Для переноса составных слов
%\XeTeXinterchartokenstate=1
\XeTeXcharclass `\- 24
\XeTeXinterchartoks 24 0 ={\hskip\z@skip}
\XeTeXinterchartoks 0 24 ={\nobreak}

%%% Ставим подпись к рисункам. Вместо «рис. 1» будет «Рисунок 1»
\addto{\captionsrussian}{\renewcommand{\figurename}{Рисунок}}
%%% Убираем точки после цифр в заголовках
\def\russian@capsformat{%
  \def\postchapter{\@aftersepkern}%
  \def\postsection{\@aftersepkern}%
  \def\postsubsection{\@aftersepkern}%
  \def\postsubsubsection{\@aftersepkern}%
  \def\postparagraph{\@aftersepkern}%
  \def\postsubparagraph{\@aftersepkern}%
}



% Автоматически переносить на след. строку слова которые не убираются
% в строке
\sloppy

%%% Для вставки рисунков
\usepackage{graphicx}

%%% Для вставки интернет ссылок, полезно в библиографии
\usepackage{url}

%%% Подподразделы(\subsubsection) не выводим в содержании
\setcounter{tocdepth}{2}

%%% Каждый раздел (\section) с новой страницы
\let\stdsection\section
\renewcommand\section{\newpage\stdsection}

%%% В введении нумерация подразделов идёт с буквой «В» (например В.1)
\makeatletter
\renewcommand\thesubsection{\ifnum\c@section=0{В.\arabic{subsection}}\else{\arabic{section}.\arabic{subsection}}\fi}
\makeatother
